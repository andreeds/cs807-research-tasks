\section{Common base with SP}
\label{sec:common}

The \emph{Simple propagation} (SP) algorithm, shown in Algorithm \ref{alg:sp}, runs in a factorization ${\cal F}$ finding all potentials $\phi(W)$ with a special property called ``one in, one out''.
Namely, given two sets of variables $X$ and $Y$, $\phi(W)$ has one variable in $X-Y$ and another variable not in $X-Y$.
SP returns a set of potentials that can be split into two disjoint subsets: one containing potentials $\phi(W)$ where $W \subseteq X$ and another ones with potentials $\phi(W)$ where $W \cap X = \emptyset$.

\begin{algorithm}[htb]
    \caption{Simple Propagation.}
    \label{alg:sp}
    \begin{algorithmic}[1]
        \Procedure{SPropagation}{${\cal F}$, $X$, $Y$}
        \While{$\exists ~ \phi(W) \in {\cal F}$ with $v \notin X-Y$ and $v^\prime \in X-Y$}\label{alg:sp_in_out}
        	\State ${\cal F}$ = \Call{RemoveBarren}{$v$, ${\cal F}$, $S$} \label{alg:sp_barren}
            \State ${\cal F}$ = \Call{SumOut}{$v$, ${\cal F}$} \label{alg:sp_sum_out}
        \EndWhile
    \Return{${\cal F}$}
    \EndProcedure
    \end{algorithmic}
\end{algorithm}

\begin{example}
Run Algorithm \ref{alg:sp} with ${\cal F} ~=~ \{P(a)$, $P(b|a)$, $P(c|a)$, $P(d|b,c)$, $P(e|d)$, $P(f|d,e)$, $P(g|e)$, $P(h|e)$, $P(i|d,g)$, $P(j|i)$, $P(k|j)$, $P(l|k)$, $P(m|h,l)\}$, $X = \{i,j,k,l,m\}$, and $Y = \{d\}$.
SP returns ${\cal F} ~=~ \{P(a)$, $P(b|a)$, $P(c|a)$, $P(n|c)$, $P(d=0|b,c)$, $P(j|i)$, $P(i,m|d=0,l)\}$.
\label{ex:sp}
\end{example}

In Example \ref{ex:sp}, it is worth noticing that the returned factorization ${\cal F}$ can be split into two disjoin subsets: 
\begin{align}
	\{ P(j|i), P(i,m|d=0,l) \}
	\label{eq:1}
\end{align}
and
\begin{align}
	\{ P(a), P(b|a)\}, P(c|a), P(c|a), P(n|c), P(d=0|b,c) \}.
	\label{eq:2}
\end{align}
(\ref{eq:1}) contains potentials with variables in $X$ and ({\ref{eq:2}}) the potentials with no variables in $X$.


\section{Unifying Inference and Modeling}
\label{sec:new}

SP can be used to perform BN inference.
The set of potentials ${\cal F}_{SP}$ returned by Algorithm \ref{alg:sp}, $\Call{SPropagation}{{\cal F},X,Y}$, contains all potentials relevant for a query $P(X|Y)$.
As shown in \cite{butzetal16}, all potentials $\phi(W) \in {\cal F}_{SP}$ where $W \subseteq X$ has $W$ d-connected (not d-separated) to $X$.

\begin{algorithm}[htb]
    \caption{Inference.}
    \label{alg:sp_inf}
    \begin{algorithmic}[1]
        \Procedure{Inference}{${\cal F}$, $X$, $Y$}
    \State ${\cal F}_{SP}$ = $\Call{SPropagation}{{\cal F}, X, Y}$
    %todo: discuss W \subseteq X or W \subseteq XY
    \State ${\cal F}^\prime$ = $\{\phi(W) ~|~ \phi(W) \in {\cal F}_{SP}$ and $W \subseteq X\}$
    \State $P(X,Y)$ = $\prod{\phi \in {\cal F}^\prime}$    
    \Return{$P(X,Y)/P(Y)$}
    \EndProcedure
    \end{algorithmic}
\end{algorithm}

\begin{example}
Given a query $P(i,j,k,l,m | d=0)$ posed to BN ${\cal B}$ in Figure \ref{fig:dag}, run Algorithm \ref{alg:sp_inf} with ${\cal F} ~=~ \{P(a)$, $P(b|a)$, $P(c|a)$, $P(d|b,c)$, $P(e|d)$, $P(f|d,e)$, $P(g|e)$, $P(h|e)$, $P(i|d,g)$, $P(j|i)$, $P(k|j)$, $P(l|k)$, $P(m|h,l)\}$, $X = \{i,j,k,l,m\}$, and $Y = \{d\}$.
\label{ex:sp_inf}
\end{example}

The important point in Example \ref{ex:sp_inf} is the use of SP to perform the bulk of the work to answer the query $P(i,j,k,l,m | d=0)$.

SP can be used to test independencies.
The set of potentials ${\cal F}_{SP}$ returned by Algorithm \ref{alg:sp}, $\Call{SPropagation}{{\cal F},X,Y}$, contains original CPTs from the BN for variables that are independent of $X$    relevant for a query $P(X|Y)$.
As shown in \cite{butzetal16}, all potentials $\phi(W) \in {\cal F}_{SP}$ where $W \cap X = \emptyset$ has $W$ d-separated from $X$.
Thus, to test the independence $I(X,Y,Z)$ we only need to check if $Z$ is a subset of the variables from the CPT $\phi(W)$ where $W \cap X = \emptyset$.

\begin{algorithm}[h]
    \caption{Modeling.}
    \label{alg:sp_mod}
    \begin{algorithmic}[1]
        \Procedure{Modeling}{${\cal F}$, $X$, $Y$, $Z$}
    \State ${\cal F}_{SP}$ = $\Call{SPropagation}{{\cal F}, X, Y}$
    \State ${\cal F}_{Rest}$ = $\{P(v_i|P_i) ~|~ P(v_i|P_i) \in {\cal F}_{SP}$ and $F_i \cap X = \emptyset\}$
    \State $\bar{R}$ = $\{v_i ~|~ P(v_i|P_i) \in {\cal F}_{Rest}\}$    
    \Return{$Z \subseteq \bar{R}$}
    \EndProcedure
    \end{algorithmic}
\end{algorithm}

\begin{example}
Given a test of independence $I(ijklm, d, a)$ posed to BN ${\cal B}$ in Figure \ref{fig:dag}, run Algorithm \ref{alg:sp_mod} with ${\cal F} ~=~ \{P(a)$, $P(b|a)$, $P(c|a)$, $P(d|b,c)$, $P(e|d)$, $P(f|d,e)$, $P(g|e)$, $P(h|e)$, $P(i|d,g)$, $P(j|i)$, $P(k|j)$, $P(l|k)$, $P(m|h,l)\}$, $X = \{i,j,k,l,m\}$, $Y = \{d\}$, and $Z = \{a\}$.
\label{ex:sp_mod}
\end{example}

The important point in Example \ref{ex:sp_mod} is the use of SP to perform the bulk of the work to test the independence $I(ijklm, d, a)$.
