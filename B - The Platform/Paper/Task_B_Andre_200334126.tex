\documentclass[twoside,11pt]{article}

%%%%% PACKAGES %%%%%%
\usepackage{pgm2016}
\usepackage{amsmath}
\usepackage{algorithm}
\usepackage[noend]{algpseudocode}
\usepackage{subcaption}
\usepackage[utf8]{inputenc}		%NOT USED?
\usepackage[english]{babel}		%NOT USED?
\usepackage{paralist}			%NOT USED?
\usepackage[lowtilde]{url}
\usepackage{fixltx2e}
\usepackage{listings}
\usepackage{color}



%%%%% MACROS %%%%%%
\algrenewcommand\Return{\State \algorithmicreturn{} }
\algnewcommand{\LineComment}[1]{\State \(\triangleright\) #1}
\renewcommand{\thesubfigure}{\roman{subfigure}}
\definecolor{codegreen}{rgb}{0,0.6,0}
\definecolor{codegray}{rgb}{0.5,0.5,0.5}
\definecolor{codepurple}{rgb}{0.58,0,0.82}
\definecolor{backcolour}{rgb}{0.95,0.95,0.92}
\lstdefinestyle{mystyle}{
   backgroundcolor=\color{backcolour},  
   commentstyle=\color{codegreen},
   keywordstyle=\color{magenta},
   numberstyle=\tiny\color{codegray},
   stringstyle=\color{codepurple},
   basicstyle=\footnotesize,
   breakatwhitespace=false,        
   breaklines=true,                
   captionpos=b,                    
   keepspaces=true,                
   numbers=left,                    
   numbersep=5pt,                  
   showspaces=false,                
   showstringspaces=false,
   showtabs=false,                  
   tabsize=2
}
\lstset{style=mystyle}

%%%%% SHORT HEADING %%%%%%
% Short headings should be running head and authors last names
\ShortHeadings{Research Task B - The Platform: Wit.ai}{dos Santos}
\firstpageno{1}

\begin{document}

\title{The Platform: Wit.ai}

\author{\name André E. dos Santos \email dossantos@cs.uregina.ca \\
\addr Department of Computer Science \\
University of Regina \\ 
Regina, Canada
}



\maketitle
%todo readme file
\begin{abstract}%   <- trailing '%' for backward compatibility of .sty file
%todo abstract
\end{abstract}


%\begin{lstlisting}[language=bash]
%mkdir folder
%\end{lstlisting}

\section{Introduction}
\label{sec:intro}

Natural language processing
claoud service
Wit
Facebook

wit funtions. How it works. what is it for

how this paper is organized


\section{Background}
\label{sec:back}

natural processing

cloud service

chrome

javascript and html

\section{Getting Started}
\label{sec:get}


In this section, we demonstrate how to build a voice application that gets colour information from user and changes one html object ``on the fly.''
Before anything else, it is needed to sign up on Wit.ai to create the voice application.
On Wit.ai home page is possible to sign in with a GitHub \cite{} id.
In fact, having a GitHub is a requirement.


%todo webkit-based browsers
\emph{webkit-based} browsers


\subsection{The Console}

Once with access to Wit.ai, one is already able to access to what is called the Wit Console. 
The Console is where we can manage the Wit.ai-powered apps, configure a voice app and improve it.
Thus, on the Wit Console App is where the voice applications logic abides.
For instance, the Wit Console with Apps  HelloWorld and ColourTest is shown in Figure \ref{} \subref{} and the Wit App for is ColourTest depicted n Figure \ref{} \subref{}
Note that if a user signs in for the first time, Wit.ai creates the first app and the user will land on its page.
We can create new apps from the top-right menu.

%todo add figure

\subsection{Determining User Intent}


steps wit learning...ect

\subsection{Going Beyond}

\section{Other Application of Wit.ai}
\label{sec:app}


\vskip 0.2in
\bibliography{references/references}
\end{document}
