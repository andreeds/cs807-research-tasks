\section{Background}
\label{sec:back}

Let $U = \{ v_1, v_2, \ldots , v_n \}$ be a finite set of variables, each with a finite domain, and $V$ be the domain of $U$.
A \emph{potential} on $V$ is a function $\phi$ such that $\phi(v) \geq 0$ for each $v \in V$, and at least one $\phi(v) > 0$.
Henceforth, we say $\phi$ is on $U$ instead of $V$.
A \emph{joint probability distribution} is a potential $P$ on $U$, denoted $P(U)$, that sums to one.
For disjoint $X,Y \subseteq U$, a \emph{conditional probability table} (CPT) $P(X|Y)$ is a potential over $X \cup Y$ that sums to one for each value $y$ of $Y$.
For simplified notation, $\{ v_1, v_2, \ldots, v_n \}$ may be written as $v_1 v_2 \cdots v_n$, and $X \cup Y$ as $XY$.

A \emph{Bayesian network} (BN) \cite{pear88} is a \emph{directed acyclic graph} (DAG) ${\cal B}$ on $U$ together with CPTs $P(v_1 | Pa(v_1))$, $P(v_2|Pa(v_2))$, $\ldots$, $P(v_n|Pa(v_n))$, where $Pa(v_i)$ denotes the parents (immediate predecessors) of $v_i$ in ${\cal B}$.
For example, Figure \ref{fig:dag} depicts a BN, where CPTs $P(a),P(b|a),\ldots,P(m|g,l)$ are not shown.
We call $\cal{B}$ a BN, if no confusion arises.
The product of the CPTs for $\cal{B}$ on $U$ is a joint probability distribution $P(U)$.

\begin{figure}[htb]
    \begin{center}
    	\includegraphics{figures/DAG.png}
		\caption{A BN extended from \cite{madsen99}.}
		\label{fig:dag}
    \end{center}
\end{figure}

\emph{Variable elimination} (VE) \cite{zhan94} computes $P(X|Y=y)$, where $X$ and $Y$ are disjoint subsets of $U$, and $Y$ is observed taking value $y$.
In VE (given as Algorithm \ref{alg:ve} below), $\Phi$ is the set of CPTs for ${\cal B}$, $X$ is a list of query variables, $Y$ is a list of observed variables, $y$ is the corresponding list of observed values, and $\sigma$ is an \emph{elimination ordering} \cite{Kja90} for variables $U - (XE)$.
All elimination orderings in this paper are determined using \emph{weighted-min-fill} (WMF), which tends to be one of the best heuristics in practice \cite{koll09}.
Evidence may not be denoted for simplified notation.

\begin{algorithm}
\caption{Variable elimination}
\label{alg:ve}
\begin{algorithmic}[1]
\Function{Variable Elimination}{$\Phi$, $X$, $Y$, $y$, $\sigma$}
    \State Delete rows disagreeing with $Y=y$ from $\phi \in \Phi$ \label{alg:ve_delete}
    \While{$\sigma$ is not empty} \label{alg:ve_while}
        \State Remove the first variable $v$ from $\sigma$
        \State $\Phi$$ ~=~$ \Call{sum-out}{$v,\Phi$}
    \EndWhile \label{alg:ve_end_while}
    \State $P(X,Y=y) ~=~ \prod_{\phi \in \Phi} \phi$ \label{alg:ve_posterior}
    \Return $P(X,Y=y) / \sum_X P(X,Y=y)$ \label{alg:ve_return}
\EndFunction
\end{algorithmic}
\end{algorithm}

VE calls the sum-out algorithm, which eliminates $v$ from a set $\Phi$ of potentials by multiplying together all potentials involving $v$ and then summing $v$ out of the product.

\begin{example}
Run Algorithm \ref{alg:ve} to compute $P(i,j,k,l,m | d=0)$ in the BN ${\cal B}$ of Figure \ref{fig:dag}.
\end{example}

\emph{d-Separation} \cite{pear88} tests independencies in DAGs and can be presented as follows \cite{darwiche09}.
Let $X$, $Y$, and $Z$ be pairwise disjoint sets of variables in a DAG ${\cal B}$.
We say $X$ and $Z$ are \emph{d-separated} by $Y$, denoted $I(X,Y,Z)$, if at least one variable on every undirected path from $X$ to $Z$ is closed.
On a path, there are three kinds of variable $v$:
\begin{inparaenum}[(i)]
	\item a \emph{sequential} variable means $v$ is a parent of one of its neighbours and a child of the other;
	\item a \emph{divergent} variable is when $v$ is a parent of both neighbours; and
	\item a \emph{convergent} variable is when $v$ is a child of both neighbours.
\end{inparaenum}
A variable $v$ is either open or closed.
A sequential or divergent variable is \emph{closed}, if $v \in Y$.
A convergent variable is \emph{closed}, if $( v \cup De(v) ) \cap Y = \emptyset$.
A path with a closed variable is \emph{blocked}; otherwise, it is \emph{active}.
[TODO] Inline example testing $I(a,d,k)$ with d-separation.

The linear implementation of d-separation given in Algorithm \ref{alg:dsep} \cite{koll09} has two phases.
Phase I determines the ancestors $An(Y)$ of $Y$ in the DAG ${\cal B}$ using the algorithm ANCESTORS (not shown).
Phase II, uses the output of Phase I to determine all variables reachable from $X$ via active paths.
This is more involved, since the algorithm must keep track of whether a variable $v$ is visited from a child, denoted $(\uparrow,v)$, or visited from a parent, denoted $(\downarrow,v)$.
In Algorithm \ref{alg:dsep}, $L$ is the set of variables to be visited, $R$ is the set of reachable variables via active paths, and $V$ is the set of variables that have been visited.

\begin{algorithm}[htb]
\caption{\cite{koll09} Find nodes reachable from $X$ given $Y$ via active paths in DAG ${\cal B}$}
\label{alg:dsep}
\begin{algorithmic}[1]
\Procedure{Reachable}{$X$,$Y$,${\cal B}$}
	\LineComment{Phase I: insert $Y$ and all ancestors of $Y$ into $A$}
	\State $An(Y) \leftarrow \Call{Ancestors}{Y,{\cal B}}$
	\State $A \leftarrow An(Y) \cup Y$
	\LineComment{Phase II: traverse active paths starting from $X$}
	\For{$v \in X$} \Comment{(Node,direction) to be visited}
		\State $L \leftarrow L \cup \{(\uparrow,v)\}$
	\EndFor
	\State $V \leftarrow \emptyset$ \Comment{(Node,direction) marked as visited}
	\State $R \leftarrow \emptyset$ \Comment{Nodes reachable via active path}
	\While{$L \neq \emptyset$} \Comment{While variables to be checked}
		\State Select $(d,v)$ in $L$ \label{alg:lin_select_dsep}
		\State $L \leftarrow L - \{(d,v)\}$
		\If{$(d,v) \notin V$}
			\If{$v \notin Y$}
				\State $R \leftarrow R \cup \{v\}$ \Comment{$v$ is reachable}
			\EndIf
			\State $V \leftarrow V \cup \{(d,v)\}$ \Comment{Mark $(d,v)$ as visited}
			\If{$d = \uparrow$ and $v \notin Y$}
				\For{$v_i \in Pa(v)$}
					\State $L \leftarrow L \cup \{(\uparrow,v_i)\}$
				\EndFor
				\For{$v_i \in Ch(v)$}
					\State $L \leftarrow L \cup \{(\downarrow,v_i)\}$
				\EndFor
			\ElsIf{$d = \downarrow$}
				\If{$v \notin Y$}
					\For{$v_i \in Ch(v)$}
						\State $L \leftarrow L \cup \{(\downarrow,v_i)\}$
					\EndFor
				\EndIf
				\If{$v \in A$}
					\For{$v_i \in Pa(v)$}
						\State $L \leftarrow L \cup \{(\uparrow,v_i)\}$
					\EndFor
				\EndIf
			\EndIf
		\EndIf
	\EndWhile
	\Return{$R$}
\EndProcedure
\end{algorithmic}
\end{algorithm}

\begin{example}
Run Algorithm \ref{alg:dsep} to test $I(a,d,k)$ in BN ${\cal B}$ illustrated in Figure \ref{fig:dag}.
\end{example}
